% !TEX program = xelatex
\documentclass[]{article}
\usepackage{commons/course}

\begin{document}
\printheader

\section*{سوال اول}
\subsection*{قسمت اول}
برای خط 14 داریم:
\begin{latin}
\centering
\begin{tabular}{|c|c|c|c|}
    \hline
    Index & Lexeme & Type & Attributes\\
    \hline
    1 & P & Program Definition & \\
    2 & a & real & \\
    3 & Q & Procedure & \\
    4 & i & int & \\
    5 & b & int array & 1-D, 10 elements \\
    6 & R & Procedure & \\
    7 & d & int & \\
    8 & a & int & \\
    9 & c & int array & 1-D, 5 elements \\
    10 & T & Procedure & \\
    11 & j & real & \\
    12 & d & real & \\
    \hline
\end{tabular}
\begin{tabular}{|c|}
    \hline
    Scope Stack\\
    \hline
    0\\
    \hline
    4\\
    \hline
    7\\
    \hline
    11\\
    \hline
\end{tabular}
\end{latin}
برای خط 19 داریم:
\begin{latin}
\centering
\begin{tabular}{|c|c|c|c|}
    \hline
    Index & Lexeme & Type & Attributes\\
    \hline
    1 & P & Program Definition & \\
    2 & a & real & \\
    3 & Q & Procedure & \\
    4 & i & int & \\
    5 & b & int array & 1-D, 10 elements \\
    6 & S & Procedure & \\
    7 & d & real & \\
    \hline
\end{tabular}
\begin{tabular}{|c|}
    \hline
    Scope Stack\\
    \hline
    0\\
    \hline
    4\\
    \hline
    7\\
    \hline
\end{tabular}
\end{latin}
\subsection*{قسمت دوم}
\begin{enumerate}
    \item در خط 6 ممکن است که به خطای \lr{index out of bounds} بر بخوریم.
    \item در خط 10 نیز دقیقا همین مشکل وجود دارد.
    \item در خط 14 نیز دقیقا همین \lr{dynamic error} ممکن است که به وجود بیاید.
    \item در خط 6 ممکن است که بخاطر جمع کردن یک real با int و ریختن آن در یک آرایه int خطای cast به وجود بیاید. (\lr{type error})
    \item مشکل ذکر شده در خط 11 نیز وجود دارد.
\end{enumerate}
\section*{سوال دوم}
می‌توان این قانون را به صورت زیر نوشت:
\begin{latin}
    \centering
    F $\rightarrow$ \textbf{do} \texttt{\#allocate\_result} E \texttt{\#true\_case} \textbf{if} E \texttt{\#condition} \textbf{else} E \texttt{\#exited} \textbf{fi}
\end{latin}
هر کدام از
\lr{action}ها
را به صورت زیر تعریف می‌کنیم:
\codesample{codes/action-definition.txt}
سعی کردم که در خود کد با کامنت توضیح بدهم که چرا هر خط نوشته شده است. همچنین در اینجا اولین
متغیر در هر دستور عملا نتیجه دستور است. (مثل سینتکس اسمبلی اینتل)

حال ورودی داده شده را پارس می‌کنیم. فرض کنید که
\lr{symbol table}
به صورت زیر است:
\begin{latin}
\centering
\begin{tabular}{|c|c|c|}
    \hline
    Lexeme & Address\\
    \hline
    a & 100\\
    b & 101\\
    e & 102\\
    c & 103\\
    d & 104\\
    h & 105\\
    \hline
\end{tabular}
\end{latin}
در ادامه برای دستور‌ها تا زمانی که به اولین
action
که خودمان اضافه کردیم داریم:
\begin{latin}
\centering
\begin{tabular}{|c|c|c|}
    \hline
    i & PB[i] & Semantic Action\\
    \hline
    0 & (+, 500 a, b) & \#add\\
    \hline
\end{tabular}
\begin{tabular}{|c|}
    \hline
    Semantic Stack (bottom to top)\\
    \hline
    501 (address of result of \textbf{do if})\\
    \hline
    500 (address of a + b)\\
    \hline
\end{tabular}
\end{latin}
حال با رسیدن به
\lr{\#true\_case}
آدرس جواب
\lr{(5 + e)}
در استک است.
\begin{latin}
\centering
\begin{tabular}{|c|c|c|}
    \hline
    i & PB[i] & Semantic Action\\
    \hline
    0 & (+, 500 a, b) & \#add\\
    1 & (+, 502, \#5, e) & \#add\\
    \hline
\end{tabular}
\begin{tabular}{|c|}
    \hline
    Semantic Stack (bottom to top)\\
    \hline
    502 (address of e + 5)\\
    \hline
    501 (address of result of \textbf{do if})\\
    \hline
    500 (address of a + b)\\
    \hline
\end{tabular}
\end{latin}
و بعد از اجرای آن داریم:
\begin{latin}
\centering
\begin{tabular}{|c|c|c|}
    \hline
    i & PB[i] & Semantic Action\\
    \hline
    0 & (+, 500 a, b) & \#add\\
    1 & (+, 502, \#5, e) & \#add\\
    2 & (:=, 501, 502) & \#true\_case\\
    \hline
\end{tabular}
\begin{tabular}{|c|}
    \hline
    Semantic Stack (bottom to top)\\
    \hline
    501 (address of result of \textbf{do if})\\
    \hline
    500 (address of a + b)\\
    \hline
\end{tabular}
\end{latin}
حال خود شرط تبدیل به کد می‌شود که باعث می‌شود که بعد از اجرای
\lr{\#condition}
به حالت زیر برسیم:
\begin{latin}
\centering
\begin{tabular}{|c|c|c|}
    \hline
    i & PB[i] & Semantic Action\\
    \hline
    0 & (+, 500 a, b) & \#add\\
    1 & (+, 502, \#5, e) & \#add\\
    2 & (:=, 501, 502) & \#true\_case\\
    3 & (+, 503, c, d) & \#add\\
    4 & ? & \#condition\\
    \hline
\end{tabular}
\begin{tabular}{|c|}
    \hline
    Semantic Stack (bottom to top)\\
    \hline
    4 (to be backpatched)\\
    \hline
    503 (address of c + d)\\
    \hline
    501 (address of result of \textbf{do if})\\
    \hline
    500 (address of a + b)\\
    \hline
\end{tabular}
\end{latin}
حال وارد یک
\textbf{do}
دیگر می‌شویم. پس داریم:
\begin{latin}
\centering
\begin{tabular}{|c|c|c|}
    \hline
    i & PB[i] & Semantic Action\\
    \hline
    0 & (+, 500 a, b) & \#add\\
    1 & (+, 502, \#5, e) & \#add\\
    2 & (:=, 501, 502) & \#true\_case\\
    3 & (+, 503, c, d) & \#add\\
    4 & ? & \#condition\\
    \hline
\end{tabular}
\begin{tabular}{|c|}
    \hline
    Semantic Stack (bottom to top)\\
    \hline
    504 (address of result of \textbf{do if})\\
    \hline
    4 (to be backpatched)\\
    \hline
    503 (address of c + d)\\
    \hline
    501 (address of result of \textbf{do if})\\
    \hline
    500 (address of a + b)\\
    \hline
\end{tabular}
\end{latin}
سپس دوباره به ترتیب داریم:
\begin{latin}
\centering
\begin{tabular}{|c|c|c|}
    \hline
    i & PB[i] & Semantic Action\\
    \hline
    0 & (+, 500 a, b) & \#add\\
    1 & (+, 502, \#5, e) & \#add\\
    2 & (:=, 501, 502) & \#true\_case\\
    3 & (+, 503, c, d) & \#add\\
    4 & ? & \#condition\\
    5 & (*, 505 3, a) & \#mult\\
    6 & (:=, 504, 505) & \#true\_case\\
    \hline
\end{tabular}
\begin{tabular}{|c|}
    \hline
    Semantic Stack (bottom to top)\\
    \hline
    504 (address of result of \textbf{do if})\\
    \hline
    4 (to be backpatched)\\
    \hline
    503 (address of c + d)\\
    \hline
    501 (address of result of \textbf{do if})\\
    \hline
    500 (address of a + b)\\
    \hline
\end{tabular}
\end{latin}
\begin{latin}
\centering
\begin{tabular}{|c|c|c|}
    \hline
    i & PB[i] & Semantic Action\\
    \hline
    0 & (+, 500 a, b) & \#add\\
    1 & (+, 502, \#5, e) & \#add\\
    2 & (:=, 501, 502) & \#true\_case\\
    3 & (+, 503, c, d) & \#add\\
    4 & ? & \#condition\\
    5 & (*, 505, \#3, a) & \#mult\\
    6 & (:=, 504, 505) & \#true\_case\\
    7 & (+, 506, \#5, h) & \#add\\
    8 & ? & \#condition\\
    \hline
\end{tabular}
\begin{tabular}{|c|}
    \hline
    Semantic Stack (bottom to top)\\
    \hline
    8 (to be backpatched)\\
    \hline
    506 (address of 5 + h)\\
    \hline
    504 (address of result of \textbf{do if})\\
    \hline
    4 (to be backpatched)\\
    \hline
    503 (address of c + d)\\
    \hline
    501 (address of result of \textbf{do if})\\
    \hline
    500 (address of a + b)\\
    \hline
\end{tabular}
\end{latin}
حال در ادامه
\lr(b + c)
تبدیل به کد می‌شود. در اینجا متغیر‌های کامپایلر به صورت زیر در می‌آید:
\begin{latin}
\centering
\begin{tabular}{|c|c|c|}
    \hline
    i & PB[i] & Semantic Action\\
    \hline
    0 & (+, 500 a, b) & \#add\\
    1 & (+, 502, \#5, e) & \#add\\
    2 & (:=, 501, 502) & \#true\_case\\
    3 & (+, 503, c, d) & \#add\\
    4 & ? & \#condition\\
    5 & (*, 505, \#3, a) & \#mult\\
    6 & (:=, 504, 505) & \#true\_case\\
    7 & (+, 506, \#5, h) & \#add\\
    8 & ? & \#condition\\
    9 & (+, 507, b, c) & \#add\\
    \hline
\end{tabular}
\begin{tabular}{|c|}
    \hline
    Semantic Stack (bottom to top)\\
    \hline
    507 (address of 5 + h)\\
    \hline
    8 (to be backpatched)\\
    \hline
    506 (address of 5 + h)\\
    \hline
    504 (address of result of \textbf{do if})\\
    \hline
    4 (to be backpatched)\\
    \hline
    503 (address of c + d)\\
    \hline
    501 (address of result of \textbf{do if})\\
    \hline
    500 (address of a + b)\\
    \hline
\end{tabular}
\end{latin}
سپس به اولین
\lr{exited}
می‌رسیم که باعث می‌شود به صورت زیر در بیاید:
\begin{latin}
\centering
\begin{tabular}{|c|c|c|}
    \hline
    i & PB[i] & Semantic Action\\
    \hline
    0 & (+, 500 a, b) & \#add\\
    1 & (+, 502, \#5, e) & \#add\\
    2 & (:=, 501, 502) & \#true\_case\\
    3 & (+, 503, c, d) & \#add\\
    4 & ? & \#condition\\
    5 & (*, 505, \#3, a) & \#mult\\
    6 & (:=, 504, 505) & \#true\_case\\
    7 & (+, 506, \#5, h) & \#add\\
    8 & (jpt, 506, 11) & \#condition\\
    9 & (+, 507, b, c) & \#add\\
    10 & (:=, 504, 507) & \#exited\\
    \hline
\end{tabular}
\begin{tabular}{|c|}
    \hline
    Semantic Stack (bottom to top)\\
    \hline
    504 (address of result of \textbf{do if})\\
    \hline
    4 (to be backpatched)\\
    \hline
    503 (address of c + d)\\
    \hline
    501 (address of result of \textbf{do if})\\
    \hline
    500 (address of a + b)\\
    \hline
\end{tabular}
\end{latin}
حال همان طور که مشخص است نتیجه در خانه‌ی 504 ذخیره شده است. حال به دومین
\lr{\#exited}
می‌رسیم (برای \textbf{\lr{do if}} بیرونی)
که باعث می‌شود تغییرات زیر اعمال بشود:
\begin{latin}
\centering
\begin{tabular}{|c|c|c|}
    \hline
    i & PB[i] & Semantic Action\\
    \hline
    0 & (+, 500 a, b) & \#add\\
    1 & (+, 502, \#5, e) & \#add\\
    2 & (:=, 501, 502) & \#true\_case\\
    3 & (+, 503, c, d) & \#add\\
    4 & (jpt, 506, 12) & \#condition\\
    5 & (*, 505, \#3, a) & \#mult\\
    6 & (:=, 504, 505) & \#true\_case\\
    7 & (+, 506, \#5, h) & \#add\\
    8 & (jpt, 506, 11) & \#condition\\
    9 & (+, 507, b, c) & \#add\\
    10 & (:=, 504, 507) & \#exited\\
    11 & (:=, 501, 504) & \#exited\\
    \hline
\end{tabular}
\begin{tabular}{|c|}
    \hline
    Semantic Stack (bottom to top)\\
    \hline
    501 (address of result of \textbf{do if})\\
    \hline
    500 (address of a + b)\\
    \hline
\end{tabular}
\end{latin}
در نهایت نیز صرفا دو ضرب دیگر اجرا می‌شود و کد نهایی به صورت زیر است:
\begin{latin}
\centering
\begin{tabular}{|c|c|c|}
    \hline
    i & PB[i] & Semantic Action\\
    \hline
    0 & (+, 500 a, b) & \#add\\
    1 & (+, 502, \#5, e) & \#add\\
    2 & (:=, 501, 502) & \#true\_case\\
    3 & (+, 503, c, d) & \#add\\
    4 & (jpt, 506, 12) & \#condition\\
    5 & (*, 505, \#3, a) & \#mult\\
    6 & (:=, 504, 505) & \#true\_case\\
    7 & (+, 506, \#5, h) & \#add\\
    8 & (jpt, 506, 11) & \#condition\\
    9 & (+, 507, b, c) & \#add\\
    10 & (:=, 504, 507) & \#exited\\
    11 & (:=, 501, 504) & \#exited\\
    12 & (*, 508, 500, 501) & \#mult\\
    13 & (+, 509, h, \#3) & \#add\\
    14 & (*, 510, 508, 509) & \#mult\\
    \hline
\end{tabular}
\begin{tabular}{|c|}
    \hline
    Semantic Stack (bottom to top)\\
    \hline
    510\\
    \hline
\end{tabular}
\end{latin}
\section*{سوال سوم}
% https://jsmachines.sourceforge.net/machines/lr1.html
در ابتدا حالت‌های
\lr{LR}
را رسم می‌کنیم.
\begin{figure}[h]
    \centering
    \includegraphics[width=1\textwidth]{figure/Q3-LR.pdf}
\end{figure}
حال برای جدول داریم:
\begin{latin}
    \centering
    \begin{tabular}{|c|c|c|c|c|c|c|c|c|c|}
        \hline
        \multirow{2}{*}{State} & \multicolumn{6}{|c|}{Action} & \multicolumn{3}{|c|}{Goto}\\
        \cline{2-10}
&a&e&d&q&b&\$&S'&S&B\\
\hline
0&s2& & &r9&s4&r9& &1&3\\
\hline
1& & & & & &acc& & & \\
\hline
2& &r9&r9& &s6& & & &5\\
\hline
3& & & &s7& &r7& & & \\
\hline
4& & & &r9&s4&r9& & &8\\
\hline
5& &s9&s10& & & & & & \\
\hline
6& &r9&r9& &s6& & & &11\\
\hline
7& &s12& & & &r3& & & \\
\hline
8& & & &r8& &r8& & & \\
\hline
9& & &s13& & & & & & \\
\hline
10& &s14& & & &r2& & & \\
\hline
11& &r8&r8& & & & & & \\
\hline
12& & & & & &r6& & & \\
\hline
13& &s15& & & &r1& & & \\
\hline
14& & & & & &r5& & & \\
\hline
15& & & & & &r4& & & \\
\hline
    \end{tabular}
\end{latin}
حال برای
\lr{parse}
کردن ورودی داده شده داریم:
\begin{latin}
    \centering
    \begin{tabular}{c|c|c}
        \textbf{Stack} & \textbf{Buffer} & \textbf{Action}\\
        \hline
        0 & abbdeqaddde\$ & Shift 2\\
        0 a 2 & bdeqaddde\$ & Shift 6\\
        0 a 2 b 6 & bdeqaddde\$ & Shift 6\\
        0 a 2 b 6 b 6 & deqaddde\$ & Reduce 9\\
        0 a 2 b 6 b 6 B 11 & deqaddde\$ & Reduce 8\\
        0 a 2 b 6 B 11 & deqaddde\$ & Reduce 8 \\
        0 a 2 B 5 & deqaddde\$ & Shift 10 \\
        0 a 2 B 5 d 10 & eqaddde\$ & Shift 14 \\
        0 a 2 B 5 d 10 e 14 & qaddde\$ & ERROR! Empty stack... \\
        0 a 2 & qaddde\$ & ERROR! Empty buffer until e, d, or b \\
        0 a 2 & ddde\$ & Reduce 9 \\
        0 a 2 B 5 & dde\$ & Shift 9 \\
        0 a 2 B 5 d 9 & de\$ & Shift 13 \\
        0 a 2 B 5 d 9 d 13 & e\$ & Shift 15 \\
        0 a 2 B 5 d 9 d 13 e 15 & \$ & Reduce 4 \\
        0 S 1 & \$ & ACCEPT \\
    \end{tabular}
\end{latin}
\section*{سوال چهارم}
بله جدول و حالت‌ها عوض می‌شود. دو حالت در جدول قسمت قبل وجود داشت که صرفا در
\lr{lookahead}ها
تفاوت داشتند:
(4 و 6) و (8 و 11)
جدول جدید به صورت زیر است:
\begin{figure}[H]
    \centering
    \includegraphics[width=1\textwidth]{figure/Q4.pdf}
\end{figure}
\section*{سوال چهارم}
در ابتدا خود
\lr{control flow graph}
را رسم می‌کنیم.
\begin{figure}[H]
    \centering
    \includegraphics[width=0.4\textwidth]{figure/Q5-01-Initial.pdf}
\end{figure}
سپس بهبود‌های
\lr{arithmetic}
را انجام می‌دهیم. در اینجا صرفا ضرب‌ها را به شیفت تبدیل می‌کنیم.
\begin{figure}[H]
    \centering
    \includegraphics[width=0.4\textwidth]{figure/Q5-02-Arithmetic.pdf}
\end{figure}
سپس ترکیب
\lr{copy propagation}
با
\lr{arithmetic}
را انجام می‌دهیم. این بدین معنی است که ممکن است که در خطی داشته باشیم که
\lr{b = a + 1}
و دقیقا در خط بعدی داریم
\lr{c = b + 1}
که در اینجا می‌توان نوشت که
\lr{c = a + 2}
(البته دقت بکنید که چیزی را حذف نمی‌کنیم. صرفا بازنویسی می‌کنیم.)
\begin{figure}[H]
    \centering
    \includegraphics[width=0.4\textwidth]{figure/Q5-03-CopyPropagation.pdf}
\end{figure}
در نهایت باید
\lr{dead code}هایی
را برداریم که متغیر سمت راست حتی یک جا سمت راست در هیچ بلاک دیگر استفاده نشده است. به عنوان 
مثال می‌توان به متغیر
$t4$
اشاره کرد.
\begin{figure}[H]
    \centering
    \includegraphics[width=0.4\textwidth]{figure/Q5-04-DeadCode.pdf}
\end{figure}
حال سعی می‌کنیم که
\lr{global optimization}ها
را انجام بدهیم. نکته‌ای که در اینجا وجود دارد این است که در کد داده شده به سادگی می‌توان
مشاهده کرد که اصلا متغیری نداریم که مقدار ثابت را به خود گرفته باشد. به عنوان مثال تنها کاندید‌های
ما متغیر‌های
$a$
است که مشخصا
$a$
در بلوک‌های نزیک 21 و قبل از وارد شدن به آن عوض می‌شود و
$t2$
اما
$t2$
بیشتر اصلا کاندید
\lr{dead code elimination}
است چرا که انگار هیچ‌جا از آن استفاده نشده است! حال برای
\lr{dead code elimination}
در ابتدا
$t2$
را مورد بررسی قرار می‌دهیم. همان طور که در نمودار زیر هم می‌بینید تنها جایی که از آن استفاده شده
در بلاک ۶ است و در صورتی که الگوریتم اسلاید‌ها را اجرا بکنیم متوجه می‌شویم که کلا تغییری نمی‌کند متغیر
$t2$
پس می‌توان آن را در بلاک زیر 6 حذف کرد.
\begin{figure}[H]
    \centering
    \includegraphics[width=0.4\textwidth]{figure/Q5-05-t2-deadcode.pdf}
\end{figure}
در ادامه نیز می‌توان دید که
$b$
نیز می‌تواند در جایی حذف بشود.
\begin{figure}[H]
    \centering
    \includegraphics[width=0.4\textwidth]{figure/Q5-06-b-deadcode.pdf}
\end{figure}
در نهایت نیز می‌توان یک
\lr{optimization}
ترکیبی زد که من حس می‌کنم کامپایلر‌های ساده نتواند این را تشخیص دهند. آن هم این است که می‌توان
متغیر
$k$
را حذف کرد. برای این کار
\lr{constant propagation}
متغیر‌های
$m$ و $k$
را بررسی می‌کنیم و متوجه می‌شویم که جفت این متغیر‌ها در طول کد ثابت هستند. مانند شکل زیر:
\begin{figure}[H]
    \centering
    \includegraphics[width=0.4\textwidth]{figure/Q5-07-k-lifetime.pdf}
\end{figure}
پس می‌توان یک بهینه‌سازی
\lr{constant folding}
انجام داد و در نهایت کد به صورت زیر شود:
\begin{figure}[H]
    \centering
    \includegraphics[width=0.4\textwidth]{figure/Q5-Final.pdf}
\end{figure}
\end{document}
