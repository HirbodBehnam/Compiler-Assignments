% !TEX program = xelatex
\documentclass[]{article}
\usepackage{commons/course}

\begin{document}
\printheader

\section*{سوال اول}
\subsection*{قسمت اول}
برای خط 14 داریم:
\begin{latin}
\centering
\begin{tabular}{|c|c|c|c|}
    \hline
    Index & Lexeme & Type & Attributes\\
    \hline
    1 & P & Program Definition & \\
    2 & a & real & \\
    3 & Q & Procedure & \\
    4 & i & int & \\
    5 & b & int array & 1-D, 10 elements \\
    6 & R & Procedure & \\
    7 & d & int & \\
    8 & a & int & \\
    9 & c & int array & 1-D, 5 elements \\
    10 & T & Procedure & \\
    11 & j & real & \\
    12 & d & real & \\
    \hline
\end{tabular}
\begin{tabular}{|c|}
    \hline
    Scope Stack\\
    \hline
    0\\
    \hline
    4\\
    \hline
    7\\
    \hline
    11\\
    \hline
\end{tabular}
\end{latin}
برای خط 19 داریم:
\begin{latin}
\centering
\begin{tabular}{|c|c|c|c|}
    \hline
    Index & Lexeme & Type & Attributes\\
    \hline
    1 & P & Program Definition & \\
    2 & a & real & \\
    3 & Q & Procedure & \\
    4 & i & int & \\
    5 & b & int array & 1-D, 10 elements \\
    6 & S & Procedure & \\
    7 & d & real & \\
    \hline
\end{tabular}
\begin{tabular}{|c|}
    \hline
    Scope Stack\\
    \hline
    0\\
    \hline
    4\\
    \hline
    7\\
    \hline
\end{tabular}
\end{latin}
\subsection*{قسمت دوم}
\begin{enumerate}
    \item در خط 6 ممکن است که به خطای \lr{index out of bounds} بر بخوریم.
    \item در خط 10 نیز دقیقا همین مشکل وجود دارد.
    \item در خط 14 نیز دقیقا همین \lr{dynamic error} ممکن است که به وجود بیاید.
    \item در خط 6 ممکن است که بخاطر جمع کردن یک real با int و ریختن آن در یک آرایه int خطای cast به وجود بیاید. (\lr{type error})
    \item مشکل ذکر شده در خط 11 نیز وجود دارد.
\end{enumerate}
\section*{سوال دوم}
می‌توان این قانون را به صورت زیر نوشت:
\begin{latin}
    \centering
    F $\rightarrow$ \textbf{do} E \texttt{\#evaluate} \textbf{if} E \texttt{\#exit} \textbf{else} E \texttt{\#exited} \textbf{fi}
\end{latin}
هر کدام از
\lr{action}ها
را به صورت زیر تعریف می‌کنیم:
\codesample{codes/action-definition.txt}
سعی کردم که در خود کد با کامنت توضیح بدهم که چرا هر خط نوشته شده است. همچنین زمانی که کامل نوشتم
تمرین را برای این که از خط
\lr{\texttt{SS[top - 2] = SS[top - 1]}}
استفاده نکنیم هم یک راه به ذهنم رسید که یک اکشن بعد از
\textbf{do}
تعریف بکنیم که یک متغیر
\lr{temp}
به ما بدهد به جای اینکه بعد از
\lr{E}
این کار را بکنیم. ولی در کل به نظر من فرقی نمی‌کند صرفا کد من کثیف‌تر است! همچنین در اینجا اولین
متغیر در هر دستور عملا نتیجه دستور است. (مثل سینتکس اسمبلی اینتل)
\end{document}
